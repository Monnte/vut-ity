\documentclass[11pt,a4paper]{article}

\usepackage[left=2cm,text={17cm, 24cm},top=3cm]{geometry}
\usepackage{times}
\usepackage[utf8]{inputenc}
\usepackage[slovak]{babel}
\usepackage{xurl}
\usepackage[breaklinks]{hyperref}
\PassOptionsToPackage{hyphens}{url}\usepackage{hyperref}
\urlstyle{same}

\begin{document}

\begin{titlepage}
\begin{center}
    {\Huge{\textsc{Vysoké učení technické v~Brně}}\\}
    {\huge{\textsc{Fakulta informačních technologií}}\\}
    \vspace{\stretch{0.382}}
    
    {\LARGE{Typografie a publikování\,--\,4. projekt}\\}
    {\Huge{Typografia a citácie}\\}
    \vspace{\stretch{0.618}}
    {\Large{\today \hfill Peter Zdravecký}}
\end{center}
\end{titlepage}

\section*{Typografia v~oblasti marketingu}

V~dnešnej dobe komercie sa stále viac ukazuje dôležitosť vhodne zvolených typografických rozhodnutí pre dosiahnutie najlepších výsledkov.


Využívaním rôznych rezov a typov písma môžeme dokonca dosiahnuť emocionálnu odozvu od čitateľa.
Tieto taktiky sú vo veľkej miere potrebné práve v~oblasti marketingu a komercie, 
kde sa pozornosť čitateľa (pozorovateľa, diváka a pod.) pretavuje na peniaze~\cite{el-erasmus}. 


Nie je preto prekvapením, že sa napríklad firma Marlboro snažila výberom fontu svojho loga 
pripomenúť Divoký Západ, a teda aj maskulinitu spojenú s~kovbojmi~\cite{thesis-donev}. 
Bola to výhodná voľba, najmä v~dobe, kedy spojenie medzi fajčením a mužskosťou 
bol najväčším lákadlom zákazníkov.
Logá firiem Coca-Cola a Ford zas zvolili ornamentálnejšie písmo, ktoré sa snaží pripomenúť 
ich historický význam, stabilitu a nostalgiu.
Naopak logo firmy Apple považovalo za vhodnejšie vybrať hladší a minimalistickejší rez písma, 
ktorým sa snažia ukázať modernosť, pokrokovosť a jednoduchosť používania~\cite{thesis-ganon}.


Okrem typografie je aj farba silným nástrojom, ktorý firmy využívajú pri reklamách vo svoj prospech.
Mnoho značiek potravín a reťazcov rýchleho občerstvenia preto využíva v~logách červenú, prípadne aj žltú farbu, 
pretože podvedomo v~zákazníkoch prebúdzajú ovplyvňujú a stimulujú chuť do jedla~\cite{el-andel}. 
Reklama môže byť zameraná aj na špecifickú skupinu obyvateľov, napríklad hračky pre deti.
V~tomto prípade je dôležité brať do úvahy aj ich schopnosť čítať, a podľa toho vybrať~vhodný rez písma, keďže deti v~skoršom školskom veku
nemusia byť schopné spracovať zložitejšie pätkové písmo~\cite{serial-cleave-etal}.


Netreba však zabúdať, že reklamou sa nemusí komunikovať len so spotrebiteľmi, ale aj s~inštitúciami (investori, zamestnanci, 
alebo spoločnosť ako taká) alebo s~inými podnikmi, a treba preto aj samotná reklama sa môže snažiť produkt predať iným spôsobom a zohľadniť iné aspekty; to všetko by sa malo zohľadniť typograficky~\cite{book-white}.


Pri písaní billboardov a veľkoplošných reklamných materiálov je potrebné veľmi precízne zohľadniť viacero atribútov 
(rozmery písmen, pomery veľkosti, atď.)~\cite{serial-legge-bigelow} a typografiu využiť čo najefektívnejšie, 
so snahou vytvoriť na zákazníka najväčší dojem a zaujať jeho pozornosť. 
V~týchto prípadoch sa taktiež často musia brať do úvahy aj priestorové, technologické, či iné obmedzenia. 
Napríklad v~prípade neónových pútačov sa typografia musí zmieriť s~faktom, že záhyby trubíc musia byť zaoblené. 
Zaujímavé sú preto aj rôzne udalosti určené k~ukazovaniu rôznych dizajnov nielen nadpisov, ale priam neónového umenia,
kde sa ukazuje, že aj takéto úzko špecifické zameranie má svoj vlastný živý mikrokozmos záujemcov. 
Pre toto neónové umenie dokonca vzniklo v~Poľsku aj Neónové múzeum~\cite{mag-eye}.


Na druhej strane, v~prípade webových stránok by človek povedal, že sa tu o~časovú tieseň nejedná. 
Avšak na internete je človek zahltený veľkým množstvom informácií vo veľmi krátkom čase. 
A~preto by mala webová stránka v~čo najmenšom čase zaujať čitateľa, koncíznym spôsobom podať základné informácie, 
pričom tie by mali byť podľa dôležitosti hierarchicky zoradené, čo sa musí nejak pretaviť aj do výslednej typografie~\cite{el-holmstrom}.

V~konečnom dôsledku však netreba zabúdať, že typografický dizajn textu by mal mať úctu 
k~svojmu obsahu, a nie naopak~\cite{book-bringhurst}. Preto sa netreba prehnane povznášať nad tým, ako tá reklama vyzerá, ale čo sa reálne snaží predať.

\newpage
\bibliography{proj4}
\bibliographystyle{czechiso}
\end{document}
