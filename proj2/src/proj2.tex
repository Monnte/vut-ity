\documentclass[11pt,twocolumn,a4paper]{article}

\usepackage[left=1.5cm,text={18cm, 25cm},top=2.5cm]{geometry}
\usepackage[IL2]{fontenc}
\usepackage[utf8]{inputenc}
\usepackage[czech]{babel}
\usepackage[unicode,hidelinks]{hyperref}
\usepackage{times}
\usepackage{amsmath}
\usepackage{amsthm}
\usepackage{amssymb}

\theoremstyle{definition}
\newtheorem{definition}{Definice}
\newtheorem{theorem}{Věta}
\newcommand{\stepover}[2]{\underset{#2}{\overset{#1}{\vdash}}}

\begin{document}

\begin{titlepage}

\begin{center}
    {\Huge{\textsc{Fakulta informačních technologií}}\\}
    {\Huge{\textsc{Vysoké učení technické v~Brně}}\\}
    \vspace{\stretch{0.38}}
    
    {\LARGE{Typografie a publikování\,--\,2. projekt \\ Sazba dokumentů a matematických výrazů}\\}
    \vspace{\stretch{0.62}}

\end{center}
{\Large {2021 \hfill Peter Zdravecký (xzdrav00)}}

\date{}

\end{titlepage}

\section*{Úvod}
V~této úloze si vyzkoušíme sazbu titulní strany, matematických vzorců, prostředí a dalších textových struktur obvyklých pro technicky zaměřené texty (například rovnice~(\ref{eq1}) nebo Definice \ref{def1} na straně \pageref{def1}). 
Rovněž si vyzkoušíme používání odkazů \verb|\ref| a \verb|\pageref|.

Na titulní straně je využito sázení nadpisu podle optického středu s~využitím zlatého řezu. Tento postup byl probírán na přednášce. Dále je použito odřádkování se zadanou relativní velikostí 0.4\,em a 0.3\,em.

V~případě, že budete potřebovat vyjádřit matematickou konstrukci nebo symbol a nebude se Vám dařit jej nalézt v~samotném \LaTeX u, doporučuji prostudovat možnosti balíku maker \AmS-\LaTeX.

\section{Matematický text}
Nejprve se podíváme na sázení matematických symbolů a~výrazů v~plynulém textu včetně sazby definic a vět s~využitím balíku \texttt{amsthm}. 
Rovněž použijeme poznámku pod čarou s~použitím příkazu \verb|\footnote|. 
Někdy je vhodné použít konstrukci \verb|\mbox{}|, která říká, že text nemá být zalomen.
\begin{definition}
\label{def1}
    Rozšířený zásobníkový automat \emph{(RZA) je definován jako sedmice tvaru $A =  (Q,\Sigma,\Gamma,\delta,q_0,Z_0,F)$, kde:}
    \renewcommand\labelitemi{$\bullet$}
    \begin{itemize}
      \item \emph{$Q$ je konečná množina \textnormal{vnitřních (řídicích) stavů},}
      \item \emph{$\Sigma$ je konečná \textnormal{vstupní abeceda},}
      \item \emph{$\Gamma$ je konečná \textnormal{zásobníková abeceda},}
      \item \emph{$\delta$ je \textnormal{přechodová funkce} $Q \times (\Sigma\cup\{\epsilon\}) \times \Gamma^*\to 2^{Q\times\Gamma^*}$,}
      \item \emph{$q_0 \in Q$ je \textnormal{počáteční stav}, $Z_0 \in \Gamma$ je \textnormal{startovací symbol zásobníku} a $F \subseteq Q$ je množina \textnormal{koncových stavů}.}
    \end{itemize}

\end{definition}

Nechť $P =  (Q,\Sigma,\Gamma,\delta,q_0,Z_0,F)$ je rozšířený zásobníkový automat. \emph{Konfigurací} nazveme trojici $(q,w,\alpha) \in Q \times \Sigma^* \times \Gamma^*$, kde $q$ je aktuální stav vnitřního řízení, $w$ je dosud nezpracovaná část vstupního řetězce a $\alpha = Z_{i_1} Z_{i_2}\ldots Z_{i_k}$ je obsah zásobníku\footnote[1]{$Z_{i_1}$ je vrchol zásobníku}.

\subsection{Podsekce obsahující větu a odkaz}
\begin{definition}
\label{def2}
    Řetězec $w$ nad abecedou $\Sigma$ je přijat RZA 
    $A$ \emph{jestliže $( q_0, w, Z_0 ) \stepover{*}{A} (q_F, \epsilon, \gamma)$ pro nějaké $\gamma \in \Gamma^*$ a
    $q_F \in F$. Množinu $L(A)=\{ w \ \mid \ w \text{ je přijat RZA } A \} \subseteq$
    $\Sigma^*$ nazýváme \textnormal{jazyk přijímaný RZA} $A$.}
\end{definition}

Nyní si vyzkoušíme sazbu vět a důkazů opět s použitím balíku \texttt{amsthm}.

\begin{theorem}
    \emph{Třída jazyků, které jsou přijímány ZA, odpovídá \textnormal{bezkontextovým jazykům}.}
\end{theorem}

\begin{proof}
    V důkaze vyjdeme z Definice \ref{def1} a \ref{def2}.
\end{proof}


\section{Rovnice a odkazy}
Složitější matematické formulace sázíme mimo plynulý text. 
Lze umístit několik výrazů na jeden řádek, ale pak je třeba tyto vhodně oddělit, například příkazem \verb|\quad|.

\begin{equation}
    \sqrt[i]{x_i^3}\quad\text{kde } x_i \text{ je } i\text{-té sudé číslo splňující}\quad x_i^{x_{i}^{i^2}+2}\leq y_i^{x_i^4} \nonumber
\end{equation}

V rovnici (\ref{eq1}) jsou využity tři typy závorek s různou explicitně definovanou velikostí.

\begin{eqnarray}
    \label{eq1}
    x&=&\bigg[{\Big\{}{\big[}a+b{\big]} * c{\Big\}}^{d} \oplus 2\bigg]^{3 / 2}\\
    y&=&\lim _{x \rightarrow \infty} \frac{\frac{1}{\log _{10} x}}{\sin ^2 x+\cos ^2 x} \nonumber 
\end{eqnarray}

V této větě vidíme, jak vypadá implicitní vysázení limity $\lim _{n \rightarrow \infty}f(n)$ v normálním odstavci textu. Podobně je to i s dalšími symboly jako $\prod_{i=1}^n 2^i$ či $\bigcap_{A\in \mathcal{B}} A$. V~případě vzorců $\lim\limits_{n \rightarrow \infty}f(n)$ a $\prod\limits_{i=1}^n 2^i$ jsme si vynutili méně úspornou sazbu příkazem \verb|\limits|.

\begin{eqnarray}
    \label{eq2}
    \int_b^a g(x)\,\mathrm{d}x & = & -\int\limits_a^b f(x)\,\mathrm{d}x
\end{eqnarray}

\section{Matice}
Pro sázení matic se velmi často používá prostředí \texttt{array} a závorky (\verb|\left|, \verb|\right|).

\begin{equation*}
    \left(
        \begin{array}{ccc} 
            a-b & \widehat{\xi+\omega} & \pi \\ 
            \vec{\mathbf{a}} & \overleftrightarrow{A C} & \hat{\beta} 
        \end{array}
    \right)
=1 \Longleftrightarrow \mathcal{Q}=\mathbb{R}
\end{equation*}

\begin{equation*}
\mathbf{A}=
\left\|
    \begin{array}{cccc}
        a_{11} & a_{12} & \ldots & a_{1 n} \\
        a_{21} & a_{22} & \ldots & a_{2 n} \\
        \vdots & \vdots & \ddots & \vdots \\
        a_{m 1} & a_{m 2} & \ldots & a_{m n}
    \end{array}
\right\|
=
\left|
    \begin{array}{cc}
    t & u\hspace{0.25em} \\
    v \hspace{0.25em} & w \\
    \end{array}
\right|
=t w\!-\!u v
\end{equation*}

Prostředí \texttt{array} lze úspěšně využít i jinde.

\begin{equation*}
\binom{n}{k} = 
\left\{
    \begin{array}{cl}
        0&\text{pro } k\ <0\text{ nebo } k\ >n \\
        \frac{n!}{k!(n-k)!}&\text{pro } 0 \leq k\ \leq n.
    \end{array}
\right.
\end{equation*}

\end{document}
